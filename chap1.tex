\chapter{Introduction}

Internet has evolved into the greatest medium of communication and data exchange that the world has ever known. Since the late 1990s, it has become a vast interconnected source of information and services widely used for commercial and personal purposes. This evolution has led to the emergence of social networking, online banking and advertising, among various other commercial and non-commercial uses. Transactions over the Internet often involve the transfer of sensitive data which attackers like to tap and exploit. For example, bank account information, medical records and passwords are routinely transferred over the network. Unfortunately, user's personal computer is a weak link in this system where personal computers typically run a large number of applications, which are rarely managed in a proper way. Single visit to a compromised web page is sufficient to infect a web browser. When a user visits such a compromised website, malicious JavaScript programs are automatically loaded with HTML code in the web browser. Execution of such malicious JavaScript can expose the personal data of the user. 

Malware is a software program designed to do malicious activities on victim's computer with the intention of extracting information and exploiting resources without his consent. Researchers developed techniques for malware detection like signature detection. To overcome the malware detection techniques, malware writers came up with different types of Malwares among which Metamorphic Malware is an advanced version. In this malware, internal structure of the malware gets changed after every execution but the overall functionality remains the same. Transcriptase is a metamorphic malware implemented in JavaScript. Execution of this script infects all the JavaScript files in the folder where the malware script is placed. As a result of this infection, a morphed version of the malware script gets attached to benign JavaScript files in the folder. Whenever this infected JavaScript gets executed in any other folder, it infects other benign JavaScript files. For each infection, malware script generates a new morphed version.

The purpose of my research is to develop a Firefox browser plugin for Metamorphic JavaScript malware detection. As JavaScript malware executes in a browser, before the page gets loaded Rhino can be used to generate an opcode sequence for the JavaScript content embedded in the webpage and then the Firefox plugin will verify the generated JavaScript's opcode sequence. If JavaScript is found to be malicious, then the plugin will restrict page load. This will provide dynamic protection from malware infecting through browser. I will be using Transcriptase.

Even though, internal code of the malware gets changed after every execution, same instructions (i.e., responsible for malware functionality) have to be used somewhere in code. So several detection techniques that work on the statistical distribution of instructions to detect the malware have been developed. Some of those techniques are Hidden Markov Model, Opcode Graph Similarity and Simple Substitution Distance detection technique. One detection technique will be chosen among these techniques based on the “accuracy" and “on the fly performance", which will be implemented in Firefox plugin. 

\section{Problem} 

To typeset text, you type whatever you want. Multiple spaces are
ignored                           when typesetting, and
the end of a line is treated as another space.
Consequently, when you are typing, you can break lines anywhere, like here
or here,
since the lines are formatted automatically when you typeset the document.
You start a new paragraph by leaving a blank line.

See how easy it is to start a new paragraph? A blank line does the trick.


\section{Proposed Solution}

Typesetting text is generally pretty easy. However, there are some special
characters that will not be typeset as you might expect. In the remainder of this
section we consider some of the most common of these
special characters. 

The backslash ``\verb+\+'' is used 
as the ``escape'' character, meaning that
whatever follows a backslash is interpreted as a macro.
For example, when \verb+\LaTeX+ is typeset, it looks like \LaTeX, which 
is a lot different from LaTeX.

To get double quotes, use two single quotes. That is, the left double quote is ``, while the right double
quote is ''. When you do it correctly, quoted text looks ``like this.''
If you use the double quote key, you will always get right-quotes, which looks "like this," and is
almost certainly not what you want.

A tilde ``\verb+~+'' is used as a ``tie,'' that is, a space is inserted, but no line break can occur.
For example, you might type Dr.~Stamp just to be sure that the line of text
does not break between Dr. and Stamp, as it otherwise might.

The percent sign is used for comments---everything following a percent sign 
on a given line is ignored when you \LaTeX\ your file. % Like this stuff here
If you want a percent sign to appear in your document, use \verb+\%+, 
which will give you this \%.

The dollar sign also has special meaning, since it is used to start and end
math formulas. To typeset a dollar sign, use \verb+\$+, like this~\$.

To force \LaTeX\ to insert a space, use a backslash followed by
a space, that is, \verb+\ +. You can put in multiple extra spaces\ \ \ \ \ \ \ if you want.

\section{Why my approach is good?}

To change fonts, enclose the text in curly brackets and give the appropriate font command.
For example, to italicize text, {\it do this}, and to get boldface, {\bf this is the ticket}.
Another useful font is {\tt this one}, which produces a typewriter-like font.

