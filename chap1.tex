\chapter{Introduction}

The arrival of the Internet has completely revolutionized our personal and professional lives. With the rapid growth of Internet infrastructure, all the market sectors, social networking services, advertising and non-commerical sectors are using this technology in their workflow. This workflow eventually increased number of Internet users worldwide. As we become more dependent on the online environment, we can see massive growth of opportunities for IT criminals to take advantage of user systems. 

Internet users often share the sensitive information like bank account details or some other personal information, over the network. As personal computers and mobile phones became an important part in most people's lives, these computers became a hub of user's personal information. In this world of ubiquitous computers and persistent threats from hackers, protecting your computer is a must. Several websites are hacked to be used as distributors of malware, to infect the visitors unknowingly with viruses and malware. Single visit to a such a hacked web page is sufficient for an intruder to get the control of user's machine.

In late 2013, one of the bank’s internal computers that are used by employees to process daily transfers and conduct bookkeeping, had been infected by malware that allowed cybercriminals to record their every move. The malicious software continuously monitored the banks activities for months, sending back video feeds and images to cyber criminals about the bank's daily routines. Then the group impersonated bank officers, not only turning on various cash machines, but also transferring millions of dollars from banks into dummy accounts set up in other countries.

Consider the fact that more than 6,600 benign websites are getting hacked every single day. These legitimate websites are turned into distributors of malware by malicious hackers. Malicious code can be injected into legitimate Javascript of a benign web page. When a user visits such a compromised website, this malicious JavaScript will be executed in the victim's web browser. Execution of such malicious JavaScript can infect victim's personal computer. Most of the times, malicious JavaScript redirects the victim's web browser to load more malicious code from a remote server. This can be achieved through several means, such as adding an HTML iframe element to a page [19]. Always cybercriminals tries to obfuscate the malicious content from detection. HTML provides very few ways to obfuscate the code such as adding an HTML iframe element to a page but the huge number of methods in JavaScript makes it easy to heavily obfuscate the malicious code into Javascript.

The purpose of my research is to develop a Firefox browser plugin for Metamorphic JavaScript malware detection. In this research, we will use the Transcriptase metamorphic JavaScript malware, which is designed to infect other JavaScript files in the same folder. During the Add-on development, we consider a variety of static detection techniques and quantify the effectiveness of each. 

This paper is organized as follows. In Chapter 2, we provide background information on metamorphic malware and with an emphasis on Transcriptase metamorphic JavaScript malware that forms the basis for the research. Also in this chapter, we cover the Rhino Javascript engine. Chapter 3 outlines the details of firefox Add-on development. Then in Chapter 4, we discuss two different static metamorphic detection techniques that we apply to detect the metamorphic JavaScript malware. In Chapter 5, we present the accuracy and performance details of Firefox Add-on using the two detection techniques. Our
experimental results for the original metamorphic JavaScript appear. Chapter 6 contains the conclusion and consideration for future work.

\section{Problem} 

To typeset text, you type whatever you want. Multiple spaces are
ignored                           when typesetting, and
the end of a line is treated as another space.
Consequently, when you are typing, you can break lines anywhere, like here
or here,
since the lines are formatted automatically when you typeset the document.
You start a new paragraph by leaving a blank line.

See how easy it is to start a new paragraph? A blank line does the trick.


\section{Proposed Solution}

Typesetting text is generally pretty easy. However, there are some special
characters that will not be typeset as you might expect. In the remainder of this
section we consider some of the most common of these
special characters. 

The backslash ``\verb+\+'' is used 
as the ``escape'' character, meaning that
whatever follows a backslash is interpreted as a macro.
For example, when \verb+\LaTeX+ is typeset, it looks like \LaTeX, which 
is a lot different from LaTeX.

To get double quotes, use two single quotes. That is, the left double quote is ``, while the right double
quote is ''. When you do it correctly, quoted text looks ``like this.''
If you use the double quote key, you will always get right-quotes, which looks "like this," and is
almost certainly not what you want.

A tilde ``\verb+~+'' is used as a ``tie,'' that is, a space is inserted, but no line break can occur.
For example, you might type Dr.~Stamp just to be sure that the line of text
does not break between Dr. and Stamp, as it otherwise might.

The percent sign is used for comments---everything following a percent sign 
on a given line is ignored when you \LaTeX\ your file. % Like this stuff here
If you want a percent sign to appear in your document, use \verb+\%+, 
which will give you this \%.

The dollar sign also has special meaning, since it is used to start and end
math formulas. To typeset a dollar sign, use \verb+\$+, like this~\$.

To force \LaTeX\ to insert a space, use a backslash followed by
a space, that is, \verb+\ +. You can put in multiple extra spaces\ \ \ \ \ \ \ if you want.

\section{Why my approach is good?}

To change fonts, enclose the text in curly brackets and give the appropriate font command.
For example, to italicize text, {\it do this}, and to get boldface, {\bf this is the ticket}.
Another useful font is {\tt this one}, which produces a typewriter-like font.

