\chapter{Implementation}

\LaTeX\ can be viewed
as a compiled programming language, in contrast to that 
nightmare known as Microsoft Word,
which can be viewed as an interpreted language. So, to typeset a
document in \LaTeX, you create a text file that has the {\tt .tex} extension.
This file includes some special
commands, known as macros. Then
you compile your {\tt .tex} file by running  \LaTeX,
which produces your typeset document, as a pdf. 

In this paper, a few basics are discussed. There are plenty of good online resource
if you need help with more advanced topics.


\section{Opcode Graph Similarity} 

To typeset text, you type whatever you want. Multiple spaces are
ignored                           when typesetting, and
the end of a line is treated as another space.
Consequently, when you are typing, you can break lines anywhere, like here
or here,
since the lines are formatted automatically when you typeset the document.
You start a new paragraph by leaving a blank line.

See how easy it is to start a new paragraph? A blank line does the trick.


