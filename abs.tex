%FIXME: Good information, though a little too much for an abstract.  Move most of this to the introduction and add a paragraph summary here instead.
The internet has evolved into the greatest medium of communication and data exchange that the world has ever known. Since the late 1990s, it has become a vast interconnected source of information and services widely used for commercial and personal purposes. This evolution has led to the emergence of social networking, online banking, and advertising, among various other commercial and non-commercial uses. Transactions over the Internet often involve the transfer of sensitive data that attackers like to tap and exploit. For example, bank account information, medical records, and passwords are routinely transferred over the network. Unfortunately, a user's personal computer is a weak link in this system where personal computers typically run a large number of applications, which are rarely managed in a proper way. A single visit to a compromised web page is sufficient to infect a web browser. When a user visits such a compromised website, malicious JavaScript programs are automatically loaded with HTML code in the web browser. Execution of such malicious JavaScript can expose the personal data of the user. 

Malware is a software program designed to do malicious activities on the victim's computer with the intention of extracting information and exploiting resources without his consent. Researchers developed techniques for malware detection like signature detection. To overcome the malware detection techniques, malware writers came up with different obfuscation techniques, such as metamorphic malware. In this type of malware, the internal structure of the malware gets changed after every execution but the overall functionality remains the same. Transcriptase is metamorphic malware implemented in JavaScript. Execution of this script infects all of the JavaScript files in the folder where the malware script is placed. As a result of this infection, a morphed version of the malware script gets attached to benign JavaScript files in the folder. Whenever this infected JavaScript gets executed in any other folder, it infects other benign JavaScript files. For each infection, the malware script generates a new morphed version.

%FIXME: Add citation for Rhino & Transcriptase
The purpose of my research is to develop a Firefox addon for detecting metamorphic JavaScript malware. As JavaScript malware executes in a browser, before the page gets loaded, the Rhino JavaScript engine can be used to generate an opcode sequence for the JavaScript content embedded in the webpage and then the Firefox addon will verify the generated JavaScript's opcode sequence. If JavaScript is found to be malicious, then the addon restricts the page load. This will provide dynamic protection from malware infecting the user's machine through the web browser. I use Transcriptase as my sample of JavaScript metamorphic malware.

%NOTE: Instructions might be totally different, but functionally equivalent
Even though the internal code of the malware gets changed after every execution, the same instructions (i.e., responsible for malware functionality) have to be used somewhere in code. So several detection techniques that work on the statistical distribution of instructions to detect the malware have been developed. Some of those techniques are hidden Markov models, opcode graph similarity, and the simple substitution distance detection technique.

%FIXME: In both the intro and the abstract we'll need to give the results of the project, even though we won't talk about them in depth until much later in the paper.

